\documentstyle[margin,line,pifont,palatino,url,tabularx]{res}

%\oddsidemargin -0.25in
%\textwidth=5.75in

\oddsidemargin -0.25in
\resumewidth=7in
\textwidth=7in
\newsectionwidth{1in}
\setlength{\sectionskip}{0.1in}

\itemsep=0in
\parsep=0in



\setlength{\topmargin}{-0.75in}
%\setlength{\topmargin}{0.25in}
\setlength{\textheight}{10.25in}


\newenvironment{list1}{
  \begin{list}{$\triangleright$}{%
      \setlength{\itemsep}{0.03in}
      \setlength{\parsep}{0in} \setlength{\parskip}{0in}
      \setlength{\topsep}{0in} \setlength{\partopsep}{0in} 
      \setlength{\leftmargin}{0.2in}}}{\end{list}}

\urlstyle{same}

\begin{document}

\name{Dr. Michael Kuhlen \vspace*{.1in}}

\begin{resume}

\vspace*{0.15in}

\section{\sc Contact Information}
\vspace*{.05in}
\begin{tabularx}{\textwidth}{Xr}
Berkeley, CA / Mountain View, CA         & \url{www.mqk.name} \\
kuhlen@gmail.com                         & \url{linkedin.com/in/}$\!$\url{mikekuhlen} \\
(831) 588-1468 \\                        & \\
\end{tabularx}

\vspace*{-0.15in}
\section{\sc Experience}
\hspace*{-0.1in}
\begin{tabularx}{1.025\textwidth}{Xr}
\textbf{Fellow, Insight Data Science}, Mountain View, CA & Aug. - Oct. 2013
\end{tabularx}
\vspace*{-0.1in}
\begin{list1}
\item Created \textit{Delay Me Not!}$\;$(www.delaymenot.info), a flight delay predictor providing ticket \\ purchasing advice.
\item Analyzed 150 million domestic flights from 1987 to 2013, stored in a MySQL database.
\item Applied a variety of machine learning regression and classification algorithms (linear and logistic regression, k-nearest-neighbors, random forests) using Python's numpy, scipy, pandas, and scikit-learn packages to model flight delay predictions.
\item Designed an interactive web front end, utilizing Flask, Twitter Bootstrap, and JavaScript, hosted on my own nginx webserver.
\end{list1}

\vspace*{-0.05in}
\hspace*{-0.1in}
\begin{tabularx}{1.025\textwidth}{Xr}
\textbf{Postdoctoral Researcher} & \\
\hspace{0.25in} \textit{Theoretical Astrophysics Center, UC Berkeley}, Berkeley, CA & 2009 - 2013 \\
\hspace{0.25in} \textit{Institute for Advanced Study}, Princeton, NJ & 2006 - 2009 \\
\end{tabularx}
%\vspace*{0.025in}
\begin{list1}
\item Performed large-scale numerical N-body simulations (on 1000's of cores on NASA's \textit{Pleiades} and NCCS's \textit{Jaguar} supercomputers) of the formation of a Milky-Way-analog galaxy (featured in the Department Of Energy's OASCR \textit{Breakthroughs 2008} report on Recent Significant Advancements in Computational Science).
\item Analyzed and visualized 20TB of numerical simulation data consisting of billions of particles per time step.
\item Studied the formation of dwarf galaxies utilizing state-of-the-art cosmological adaptive mesh refinement (AMR) hydrodynamics simulations.
\item Developed C and Python tools (often MPI-parallelized) for numerical data analysis and \\ visualization.
\item Contributed to the development of the \textit{Enzo} cosmological hydrodynamics community code (\url{enzo-project.org}) written in C++ and Fortran, and the \textit{yt Project} (\url{yt-project.org}), an astrophysics data analysis and visualization package for Python.
\item Published 41 papers (18 first author) in peer reviewed journals (including Nature and Science), which together have received more than 2,500 citations.
\end{list1}

\vspace*{-0.05in}
\hspace*{-0.1in}
\begin{tabularx}{1.025\textwidth}{Xr}
\textbf{Ph.D. Student, UC Santa Cruz}, Santa Cruz, CA & 2000 - 2006
%\hspace{0.25in} \textit{UC Santa Cruz}, Santa Cruz, CA & 2000 - 2006
\end{tabularx}
\vspace*{-0.1in}
\begin{list1}

\item Applied different numerical techniques and simulation codes to study a wide range of astrophysical research topics, ranging from stellar convection to cosmological structure formation.
\item Analyzed, visualized, and synthesized simulation data with self-developed, open-source (VisIt), and commercial (IDL, Mathematica) tools.
\item Co-taught the Akamai Maui Short Course, applying inquiry-based science education techniques to prepare Hawaiian undergraduates for summer internships at local technology companies.
\end{list1}



\section{\sc Skills}
\textbf{Languages:} Python, C, MySQL, bash, HTML/CSS, \LaTeX, C++ (some exp.), JavaScript (some exp.) \vspace{0.05in} \\
\textbf{Tools:} git, hg, sed, awk, numpy, scipy, pandas, scikit-learn, matplotlib, mpi4py, IPython notebook, HDF5, Flask, Twitter Bootstrap \vspace{0.05in} \\
\textbf{Other:} Linux (10+ years), numerical simulation (N-body and AMR CFD), numerical analysis, parallel computation (MPI), visualization, machine learning and classification (some exp.) 

\section{\sc Education}
\hspace*{-0.1in}
\begin{tabularx}{1.025\textwidth}{Xr}
\textbf{University of California at Santa Cruz}, Santa Cruz, California & \\
\hspace{0.25in} \textit{Ph.D., Astronomy \& Astrophysics, ``Adventures in Numerical Astrophysics''} & June 2006 \\[0.1in]
\textbf{California Institute of Technology}, Pasadena, California & \\
\hspace{0.25in} \textit{B.S., Physics} & June 2000 \\
\end{tabularx}

\section{\sc Public Talks}
\begin{list1}
\item Mt.~Tamalpais State Park Astronomy Program (co-sponsored by Bay Area Wonderfest) \\ ``Dark Matter, Dark Skies, Bright Minds'', June 2012

\item SF Amateur Astronomers, ``The Milky Way as a Dark Matter Laboratory'', May 2012

\item ``What Physicists Do'' lecture at Sonoma State University, October 4, 2010
\end{list1}

\end{resume}
\end{document}

